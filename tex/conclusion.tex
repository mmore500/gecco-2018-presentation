\section{Conclusion}

\begin{frame}{Automap: Challenges (\& Solutions?)}

\textbf{Manual G-P map design is difficult and highly heuristic\dots but deep
learning implementation is too!}
\pause
\vspace{-1ex}
\begin{itemize}[<+->]
\itemsep0em
\item learn better quality G-P maps than manually designed?
\end{itemize}
\vspace{-1ex}
\pause
\textbf{If you already need good solutions to train, why use automap?}
\pause
\vspace{-1ex}
\begin{itemize}[<+->]
\itemsep0em
\item might yield even better solutions
\item perhaps try iterative bootstrapping of G-P autoencoder maps in absence of good solutions
\end{itemize}
\vspace{-1ex}
\pause
\textbf{Parts of phenotype space become totally inaccessible.}
\pause
\vspace{-1ex}
\begin{itemize}[<+->]
\itemsep0em
\item inherent concern with indirect G-P maps \cite{clune2008generative}
\item hopefully inaccessible parts aren't interesting
\item perhaps use HybrID approach \cite{clune2009hybrid}
\end{itemize}

\end{frame}

\begin{frame}{Automap: Next Steps}

\begin{itemize}[<+->]
\item demonstrate automap on more challenging problem(s)
\begin{itemize}
\item move beyond proof of concept\dots
\item try to show effects on evolvability \& solution quality
\item soft-bodied robots (?)
\end{itemize}
\item explore using generative adversarial networks (GANs) in place of autoencoders
\item put in conversation with learning theory \& evolvability synthesis \cite{kouvaris2017evolution}
\end{itemize}

\end{frame}
