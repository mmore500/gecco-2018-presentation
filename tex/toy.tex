\section{Experiment: Toy Problem}

\begin{frame}{Problem Definition}

\textbf{phenotype:} vector $\vec{x}$ of 100 float values
\begin{itemize}
\item leg lengths of 100-legged ``table''
\end{itemize}

\vspace{2ex}
\pause

\textbf{fitness:} $-\sigma(\vec{x})$
\begin{itemize}
\item best fitness = level table
\end{itemize}

\vspace{2ex}
\pause

\textbf{rugged fitness landscape:}
any level table is a local fitness peak

\end{frame}

\begin{frame}{Preparing Autoencoders}

\Large

\begin{itemize}[<+->]
\item make training data
\pause
\begin{itemize}[<+->]
\item evolved 250 independent 300-individual populations
\item using direct encoding
\end{itemize}
\vspace{1ex}
\item train autoencoders
\pause
\begin{itemize}[<+->]
\item bottleneck learns to map one value to all leg lengths
\item denoising learns to set leg lengths to mean leg length
\end{itemize}
\end{itemize}

\end{frame}

\begin{frame}{Experimental Procedure: Response to Selective Pressure}

\vspace{2ex}

\textbf{fitness:}
$-\sigma(⃗\vec{x}) \textcolor{h2}{- |\mu(\vec{x})/10 |}$

\vspace{2ex}

\textbf{intuition:}
level-ness still essential, but short tables tend to be favored

\vspace{2ex}
\pause

\textbf{question:}
If we start with populations of tall tables, how well can short tables evolve under different G-P maps?

\end{frame}


\begin{frame}{Result: Response to Selective Pressure}

\begin{figure}
  \includegraphics[width=0.75\linewidth]<1>{img/results/zero_leg_selection-1}
  \includegraphics[width=0.75\linewidth]<2>{img/results/zero_leg_selection-2}
  \includegraphics[width=0.75\linewidth]<3>{img/results/zero_leg_selection-3}
  \caption{
    Response to short-table selection pressure under different genotype-phenotype maps.
    Bootstrapped 95\% confidence intervals are shaded along each curve.
  }\label{fig:select_response}
\end{figure}


\end{frame}
